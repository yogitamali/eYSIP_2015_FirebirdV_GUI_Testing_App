\documentclass{article}
\raggedright
\usepackage{color}
\usepackage{graphicx}
\usepackage[a4paper,hmargin=25mm,bmargin=30mm,top=20mm]{geometry}
\begin{document}
	\begin{center}
		\Huge \color{blue}\underline{PROJECT NAME:} \vspace{4cm} \\
		\huge \color{green} GUI DEVELOPMENT FOR FIREBIRD USING- JAVA \vspace{4cm} \\
		\color{black}\Large Intern: Apoorva Bhargava \vspace{3cm} \\
		\Large Mentors: Sachin Patil \\
		\Large \qquad \qquad \qquad \quad Saurav Shandilya \\
		\Large \qquad \qquad \qquad Amiraj Dhawan 
	\end{center}
		\newpage
	\begin{center}
		\Huge \color{red} \textbf{\underline{DEVELOPER'S MANUAL}}
	\end{center}
	\vspace{1cm}
	The execution of the program starts from the main(String args[]) function of main class. The main function creates and displays the JFrame form the default constructor of the main class is called. Within the constructor initComponents() function is called. The initComponents() function initializes all the JTextField, JLabel, Jbutton, JProgressBar anf JSlider and adds them to the JFrame form. It also sets the border, foreground, background, font etc of the various GUI components and assigns ActionListener and ChangeListener to the buttons and sliders respectively. \newline \\ 
    After the execution of initComponents() method, object of SerialPortConnection class is made then the listSerialPorts() method is called which lists all the available ports in the JComboBox. All the buttons except the Connect button are disabled by using setEnabled(boolean) function and remains disable till the connect button is pressed. To make a successful connection, user needs to select a correct COM port from JComboBox and click on the connect button is clicked then the connect(portName) method of SerialPortConnection class which tries to connect the selected COM port with the GUI. If connection is successful then button of the GUI becomes enabled and ReadThread is started which writes "T" on the output stream using writeOnTerminal(String) method so as to tell robot that the sensors value needs to be sent by it. When the data ia available in the input stream serialEvent method of SerialEventHandler class implementing SerialPortEventListener interface is called. \newline \\ 
    After the successful connection user click on any button and the action associated with that button is performed  by writing on the output stream. If the connection is unsuccessful corresponding exception is displayed on the screen. \newline \\ 
    To disconnect from the robot user clicks on disconnect button. After this the ReadThread is stopped and removeSerialPorts() and removeGUIComponents() is called to close the serial port and removes the reading of sensors.\\
    \vspace{1cm}
	{\huge Methods} \vspace{1cm}
	\begin{enumerate}
		\item \textbf{jButtonBuzzerActionPerformed()} \vspace{0.25cm} \\
		Input: None\\
		Return: void\\
		Description: Checks the current status of  the buzzer stored in the variable statusBuzzer. If statusBuzzer is 0 then it writes "7" on the terminal to turn ON the buzzer and if statusBuzzer is 1 then it writes 9 on the terminal to turn OFF the buzzer and correspondingly updates the statusBuzzer.\vspace{0.5cm}\\
		\item \textbf{jButtonForwardMotionActionPerformed()} \vspace{0.25cm} \\
		Input: None\\
		Return: void\\
		Description: Makes the robot move forward by writing "8" on the output stream. \vspace{0.5cm} \\
		\item \textbf{jButtonRightMotionActionPerformed()} \vspace{0.25cm}\\
		Input: None\\
		Return: void\\
		Description: Makes the robot move right by writing "4" on the output stream. \vspace{0.5cm} \\
		\newpage
		\item \textbf{jButtonBackwardMotionActionPerformed()} \vspace{0.25cm} \\
		Input: None\\
		Return: void\\
		Description: Makes the robot move backward by writing "2" on the output stream. \vspace{0.5cm} \\
		\item \textbf{jButtonLeftMotionActionPerformed()} \vspace{0.25cm} \\
		Input: None\\
		Return: void\\
		Description: Makes the robot move left by writing "6" on the output stream. \vspace{0.5cm} \\
		\item \textbf{jButtonStopMotionActionPerformed()} \vspace{0.25cm} \\
		Input: None\\
		Return: void\\
		Description: Makes the robot stop by writing "5" on the output stream. \vspace{0.5cm} \\
		\item \textbf{jButtonCOMConnectActionPerformed()} \vspace{0.25cm} \\
		Input: None\\
		Return: void\\
		Description: Gets the selected COM port from the JComboBox by using getSelectedItem() function and calls the connect(String portName) function from SerialPortConnection class and passes the selected port as a parameter to make connection. \vspace{0.5cm} \\
		\item \textbf{jButtonCOMDisconnectActionPerformed()} \vspace{0.25cm} \\
		Input: None \\
		Return: void\\
		Description: Calls removeGUIComponents() method to disable all the enabled GUI components, calls removeSerialPorts() method to close the output and input stream and closes the serial port. It also enables the connect button by calling setEnabled() function \vspace{0.5cm} \\
		\item \textbf{getSliderValueLeftMotor()} \vspace{0.25cm} \\
		Input: None\\
		Return: void\\
		Description: Gets the changed value of slider by calling the getValue() function and then sets the value of the label by calling setText() function. \vspace{0.5cm} \\
		\item \textbf{jTextFieldLeftMotorActionPerformed()} \vspace{0.25cm} \\
		Input: None\\
		Return: void\\
		Description: Gets the value entered in the jTextFieldLeftMotor by calling getText() method and then sets the slider according to that value by calling setValue() method on jSliderLeftMotor. \vspace{0.5cm} \\
		\item \textbf{getSliderValueRightMotor()} \vspace{0.25cm} \\
		Input: None\\
		Return: void\\
		Description: Gets the changed value of slider by calling the getValue() function and then sets the value of the label by calling
		setText() function. \vspace{0.5cm} \\
		\item \textbf{jTextFieldRightMotorActionPerformed()} \vspace{0.25cm} \\
		Input: None\\
		Return: void\\
		Description: Gets the value entered in the jTextFieldRightMotor by calling getText() method and then sets the slider value by calling setValue() method on jSliderRightMotor \vspace{0.5cm} \\
		\item \textbf{jButtonSetVelocityActionPerformed()} \vspace{0.25cm} \\
		Input: None\\
		Return: void\\
		Description: Gets the left and right motor velocity from their respective TextFields and performs a check whether any of the velocity is left blank. If so, then prompts the user to set a valid velocity. Once a valid velocity has been entered then it writes "R" on the outputstream to indicate that the velocity of the robot has to change and then writes both the velocities on the outputstream. \vspace{0.5cm} \\
		\item \textbf{jButtonResetVelocityActionPerformed()} \vspace{0.25cm} \\
		Input: None\\
		Return: void\\
		Description: Resets the velocity of left and right motors of robot by writing "S" on the output stream and then sending the initial velocity of robot, i.e., 255.\vspace{0.5cm} \\
		\item \textbf{getSliderValueServo1()} \vspace{0.25cm} \\
		Input: None\\
		Return: void\\
		Description: Gets the value of angle by which the servo motor s1 has to be rotated by calling the method getValue() of JSlider class and sets the text field to that value using setText() method of JTextField class \vspace{0.5cm} \\
		\item \textbf{getSliderValueServo2()} \vspace{0.25cm} \\
		Input: None\\
		Return: void\\
		Description: Gets the value of angle by which the servo motor s2 has to be rotated by calling the method getValue() of JSlider class and sets the text field to that value using setText() method of JTextField class \vspace{0.5cm} \\
		\item \textbf{getSliderValueServo3()} \vspace{0.25cm} \\
		Input: None\\
		Return: void\\
		Description: Gets the value of angle by which the servo motor s3 has to be rotated by calling the method getValue() of JSlider class and sets the text field to that value using setText() method of JTextField class \vspace{0.5cm} \\
		\item \textbf{jButtonForwardMovementActionPerformed()} \vspace{0.25cm} \\
		Input: None\\
		Return: void\\
		Description: Sends "U" which is 0x54 in hex to the robot to identify next values are for forward movement by some distance. Gets the value of the distance by which robot has to move forward from the jTextFieldDistance by calling getText() method. The value obtained is divided and modulo by 255 to send the values greater than 255.  \vspace{0.5cm} \\    
		\item \textbf{jButtonBackwardMovementActionPerformed()} \vspace{0.25cm} \\
		Input: None \\
		Return: void\\
		Description: Sends "V" which is 0x55 in hex to the robot to identify next values are for backward movement by specified distance. Gets the value of the distance by which robot has to move forward from the jTextFieldDistance by calling getText() method. The value obtained is divided and modulo by 255 to send the values greater than 255. \vspace{0.5cm} \\
		\item \textbf{jButtonRightRotationActionPerformed()} \vspace{0.25cm} \\
		Input: None \\
		Return: \\
		Description: Sends "W" which is 0x56 in hex to the robot to identify next values are for right rotation of robot by specified angel. Gets the value of the angel by which the robot has to rotate from the jTextFieldDistance by calling getText() method. The value obtained is divided and modulo by 255 to send the values greater than 255. \vspace{0.5cm} \\
		\item \textbf{jButtonLeftRotationActionPerformed()} \vspace{0.25cm} \\
		Input: None\\
		Return: void\\
		Description: Sends "X" which is 0x57 in hex to the robot to identify next values are for right rotation of robot by specified angel. Gets the value of the angel by which the robot has to rotate from the jTextFieldDistance by calling getText() method. The value obtained is divided and modulo by 255 to send the values greater than 255. \vspace{0.5cm} \\
		\item \textbf{jButtonServo1ActionPerformed()} \vspace{0.25cm} \\
		Input: None\\
		Return: void\\
		Description: Sends 0x80 to identify that the next value is angel to be rotated by servo motor 1. Gets the value of the angle by which motor has to be rotated from the jTextFieldServo1 by calling getText method and then sends the value to the outputstream to rotate the motor. \\
		\item \textbf{jButtonServo2ActionPerformed()} \vspace{0.25cm} \\
		Input: None\\
		Return: void\\
		Description: Sends 0x81 to identify that the next value is angel to be rotated by servo motor 2. Gets the value of the angle by which motor has to be rotated from the jTextFieldServo2 by calling getText method and then sends the value to the outputstream to rotate the motor. \vspace{0.5cm} \\
		\item \textbf{jButtonServo3ActionPerformed()} \vspace{0.25cm} \\
		Input: None\\
		Return: void\\
		Description: Sends 0x82 to identify that the next value is angel to be rotated by servo motor 3. Gets the value of the angle by which motor has to be rotated from the jTextFieldServo3 by calling getText method and then sends the value to the outputstream to rotate the motor. \vspace{0.5cm} \\
		\item \textbf{jButtonLCDPrintActionPerformed()} \vspace{0.25cm} \\
		Input: None\\
		Return: void\\
		Description: Sends 0x83 to identify that next 3 bytes are to print on LCD. Gets the row no, column no and the character to be printed on the LCD screen by calling getText method and then sends these values to the output stream \vspace{0.5cm} \\
		\item \textbf{jButtonBarGraphLedActionPerformed()} \vspace{0.25cm} \\
		Input: None \\
		Return: void \\
		Description: Sends 0x84 to identify that next value is to glow specified bar graph LED. Gets the number of bar graph led to be glown from the jTextFieldBarGraphLed by calling getText method and then writes the corresponding hex value on the output stream \vspace{0.5cm} \\
		\item \textbf{jTextFieldServo1ActionPerformed()} \vspace{0.25cm} \\
		Input: None \\
		Return: void \\
		Description: Gets the value from the jTextFieldServo1 by calling getText method, then converts it into an integer value and finally sets the value of slider by calling setValue() method. \vspace{0.5cm} \\
		\item \textbf{jTextFieldServo2ActionPerformed()} \\
		Input: None \\
		Return: void \\
		Description:  Gets the value from the jTextFieldServo2 by calling getText method, then converts it into an integer value and finally sets the value of slider by calling setValue() method. \vspace{0.5cm} \\
		\item \textbf{jTextFieldServo3ActionPerformed()} \\
		Input: None \\
		Return: void \\
		Description: Gets the value from the jTextFieldServo2 by calling getText method, then converts it into an integer value and finally sets the value of slider by calling setValue() method. \vspace{0.5cm} \\
		\item \textbf{listSerialPorts()} \vspace{0.25cm} \\
		Input: None \\
		Return: void \\
		Description: List all the available ports in JComboBox. Calls removeEventListener() method of SerialPort class to stop the event that occurs while reading from terminal and also closes the serial port, input and output stream \vspace{0.5cm} \\
		\item \textbf{removeGUIComponents()} \vspace{0.25cm} \\
		Input: None \\
		Return: void \\
		Description: enable disable various components of GUI and removes text from JTextFiel and jLabel and set the value of all the jProgressBar to 0. \vspace{0.5cm} \\
		\item \textbf{connectGUIComponents()} \\
		Input: None \\
		Return: void \\
		Description: Enable all the jTextField and jLabels and starts the ReadThread to start reading the sensors value. \vspace{0.5cm} \\
		\item \textbf{connect(String)} \vspace{0.25cm} \\
		Input: String \\
		Return: void \\
		Description: Connects GUI with the selected COM port. Calls connectToPort(portName) method of SerialPortConnection class in order to connect to the robot. If connection is not possible due to some exception then displays the corresponding exception message in a dialog box. \vspace{0.5cm} \\
		\item \textbf{setSerialEventHandler()} \vspace{0.25cm} \\
		Input: None \\
		Return: void \\
		Description: Sets the serial event handler by adding the event listener.\\ addEventListener() and notifyOnDataAvailable() method of serialport class is called to set the serial event handler. \vspace{0.5cm} \\
		\item \textbf{Sharp\_Distance\_Sensor\_estimation (int value)} \vspace{0.25cm} \\
		Input: int\\
		Return: int\\
		Description: Converts the analog value of sharp IR sensor into distance in mm by using the formula:\\ distance =  10.00*(2799.6*(1.00/(Math.pow(value,1.1546)))) \vspace{0.5cm} \\
		\item \textbf{setIcon()} \vspace{0.25cm} \\
		Input: None\\
		Return: void\\
		Description: Sets the icon of GUI by calling setIconImage(Image) method \vspace{0.5cm} \\
		\item \textbf{removeSerialPorts()} \vspace{0.25cm} \\
		Input: None\\
		Return: void\\
		Description: Calls removeEventListener() method of SerialPort class to stop the event that occurs while reading from terminal and also closes the serial port, input and output stream \vspace{0.5cm} \\
		\item \textbf{serialPorts()} \vspace{0.25cm} \\
		Input: None\\
		Return: string[]\\
		Description: Ports are enumerated using getPortIdentifiers() method of CommPortIdentifier class and stored in an array using ArrayList and then converted into String array and returned to the calling function \vspace{0.5cm} \\
		\item \textbf{writeOnTerminal(String serialMessage)} \vspace{0.25cm} \\
		Input: String\\
		Return: void\\
		Description: Converts the serialMessage into bytes by using the function getBytes() and then writes those bytes on the outputstream by using the function write(bytes). It then calls the function flush() to empty the outputstream. \vspace{0.5cm} \\
		\newpage
		\item \textbf{connectToPort(String portName)} \vspace{0.25cm} \\
		Input: String- the string is the name of the port which the user selects from JComboBox\\
		Return: int - 1 if successful connection has been made\\
		              \qquad \qquad \qquad 2 if UnsupportedCommOperationException is thrown\\
		              \qquad \qquad \qquad 3 if PortInUseException is thrown\\
		               \qquad \qquad \qquad 4 if NoSuchPortException is thrown \\
		Description: Calls the getPortIdentifier(portName) method of CommPortIdentifier class to obtain a CommPortIdentifier object and then open the communication channel of that port by using open(String, int) method of CommPortIdentifier class. After this it sets the baud rate, databits, stopbits and parity of the selected port. The function -\\
		Returns 1 if successful connection has been made\\
		Returns 2 if UnsupportedCommOperationException is thrown\\
		Returns 3 if PortInUseException is thrown\\
		Returns 4 if NoSuchPortException is thrown \vspace{0.5cm}\\
		\item \textbf{setInputOutputStream()} \vspace{0.25cm} \\
		Input: None \\
		Return: void\\
		Description: Gets the input stream and output stream of the connected COM port by calling the function getInputStream() and getOutputStream(). \vspace{0.5cm} \\
	\end{enumerate}
\end{document}